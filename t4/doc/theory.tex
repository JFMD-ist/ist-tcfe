\subsection{Operating Point Analysis}
\hspace{12pt} We begin the theoretical analysis by computing the Operating Point of the transistors, resulting in the static voltages and currents. To do so, we begin by analyzing the Gain Stage. We use voltage dividers, we apply KVL on all essential meshes and also simplify the bias circuit using Thévenin's equivalent. Because we are computing the direct current only, the capacitors then become open circuits, and we get the following equations presented in the classes:

\small
\begin{gather}
    V_{eq}=V_{CC}\frac{R_{B1}}{R_{B1}+R_{B2}} \nonumber \\
    \nonumber \\ 
    R_{B}=\frac{1}{\frac{1}{R_{B1}}+\frac{1}{R_{B2}}} \nonumber \\
    \nonumber \\
    I_{B1}=\frac{V_{eq}-V_{ON}}{R_B+(1+\beta _F)R_E} \nonumber \\
    \nonumber \\
    I_{C1}=\beta _FI_B \nonumber \\
    \nonumber \\
    V_{O1}=V_{CC}-R_{C1}I_{C1} \nonumber \\
    \nonumber \\
    I_{E1}=(1+\beta _F)I_{B1} \nonumber \\
    \nonumber \\
    V_{E1}=R_{E1}I_{E1} \nonumber \\
    \nonumber \\
    V_{CE}=V_{O1}-V_{E1} \nonumber
\end{gather}
\normalsize

We have now the equations to calculate the values of the static voltages and currents for the Gain Stage.

\begin{figure}[h]
	\centering
	\input{gs_theory.tex}
	\caption{OP results for the gain stage obtained from the Octave script}
	\label{fig:gs_results_th}
\end{figure}
\pagebreak
 
Now we repeat the same process for the Output Stage, knowing that the input voltage will be the output voltage of the Gain Stage ($V_{I2}=V_{O1}$):
\small
\begin{gather}
	I_{E2}=\frac{V_{CC}-V_{ON}-V_{I2}}{R_{E2}} \nonumber \\
    	\nonumber \\
     	I_{C2}=I_{E2}\frac{\beta _F}{1+\beta _F} \nonumber \\
     	\nonumber \\
     	V_{O2}=V_{CC}-R_{E2}I_{E2} \nonumber
\end{gather}
\normalsize
\vspace{15pt}

\begin{figure}[h]
	\centering
	\input{os_theory.tex}
	\caption{OP results for the output stage obtained from the Octave script}
	\label{fig:os_results_th}
\end{figure}

\subsection{Gain and Impedance}
 
In this section we will calculate the gain and input and output impedances, beginning with the Gain Stage. Using KVL we get the following equations:
 
\vspace{15pt}
\small
\begin{gather}
	\frac{v_o}{v_i}=R_{C1}\frac{R_{E1}-g_{m1}r_{\pi 1}r_{o1}}{(r_{o1}+R_{C1}+R_{E1})(R_B+r_{\pi 1}+R_{E1})+g_{m1}R_{E1}r_{o1}r_{\pi 1}-R_{E1}^{2}} \\
	\nonumber \\
	\frac{v_o}{v_i}=\frac{g_{m1}R_{C1}}{1+g_{m1}R_{E1}}
\end{gather}
\normalsize
\vspace{15pt}
 
We can also calculate the input and output impedances by determining the equivalent impedance seen by the input voltage source and the collector pin of the NPN transistor, respectively:
\small
\begin{gather}
	Z_{I1}=\frac{(r_{o1}+R_{C1}+R_{E1})(R_B+r_{\pi 1}+R_{E1})+g_{m1}R_{E1}r_{o1}r_{\pi 1}-R_{E1}^{2}}{r_{o1}+R_{C1}+R_{E1}} \nonumber \\
    	\nonumber \\
    	Z_X=r_{o1}\frac{\frac{(R_B+r_{\pi 1})R_{E1}}{R_B+r_{\pi 1}+R_{E1}}}{\frac{1}{\frac{1}{r_{o1}}+\frac{1}{r_{\pi 1}+R_B}+\frac{1}{R_{E1}}+\frac{g_{m1}r_{\pi 1}}{r_{\pi 1}+R_B}}} \nonumber \\
   	\nonumber \\
    	Z_{O1}=\frac{1}{\frac{1}{Z_X}+\frac{1}{R_{C1}}} \nonumber
\end{gather}
\normalsize

Where $Z_{I1}$ and $Z_{O1}$ are the input and output impedances of the Gain Stage, respectively.
\vspace{15pt}

Using the same process we can calculate the gain, input impedance and output impedance of the Output Stage, and we get:
\small
\begin{gather}
    	\frac{v_o}{v_i}=\frac{g_{m2}}{g_{m2}+g_{o2}+g_{\pi 2}+g_{E2}} \nonumber \\
    	\nonumber \\
    	Z_{I2}=\frac{g_{\pi 2}+g_{E2}+g_{o2}+g_{m2}}{g_{\pi 2}(g_{\pi 2}+g_{E2}+g_{o2}} \nonumber \\
    	\nonumber \\
    	Z_{O2}=\frac{1}{g_{\pi 2}+g_{E2}+g_{o2}+g_{m2}} \nonumber
\end{gather}
\normalsize
\vspace{15pt}

Using Octave to calculate these equations, we get the following values:
\vspace{15pt}

\begin{figure}[h]
	\centering
	\input{theory_sec2.tex}
	\caption{Gain and impedance values obtained by the octave script}
	\label{fig:results_th}
\end{figure}

\vspace{15pt}

% Insert comment on the values

\vspace{15pt}

In this final section of the theoretical analysis we will compute the frequency response of the circuit. The results were joined together in one equation:
\vspace{15pt}

\fontsize{8pt}{12pt}\selectfont
\begin{gather}
    	\frac{V_o(f)}{V_i(f)}=\frac{1}{1+2\pi fjC_1R_S}\frac{-R_C(-g_{m1}r_{o1}r_{\pi 1}+Z_{eq})}{Z_{eq}(-g_{m1}r_{o1}r_{\pi 1}+Z_{eq})+(R_C+r_{o1}+Z_{eq})(-R_B-r_{\pi 1}-Z_{eq})}\frac{g_{m2}}{g_{m2}+g_{o2}+g_{\pi 2}+g_{E2}}\frac{2\pi fjC_{out}R_l}{1+2\pi fjC_{out}R_l} \nonumber
\end{gather}
\normalsize
\pagebreak

Finally we can use Octave to plot the previous equation using a logarithmic scale:

\begin{figure}[h!]
	\centering
	\subfigure[]{\includegraphics[width=0.49\textwidth, trim={0, 6cm, 0, 6cm}, clip]{theory_mag.pdf}}
	\subfigure[]{\includegraphics[width=0.49\textwidth, trim={0, 6cm, 0, 6cm}, clip]{theory_ph.pdf}}
	\caption{(a) Magnitude plot (b) Phase plot of the frequency response output}
	\label{fig:freq_th}
\end{figure}

We can define the cutoff frequencies as the ones at which the magnitude of the gain of the circuit is 3dB below the peak of the graph (which happens around 3kHz), in figure \ref{fig:freq_th} we can see the cutoff line and the 2 frequencies at which it intersects the graph, these are our values for the cutoff frequencies. The \textbf{bandwidth} of the circuit is the difference between the high and low cutoff frequencies (this can be thought of as the range of frequencies under which the circuit will amplify the input signal).

