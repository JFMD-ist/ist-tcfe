We will now analyse the natural solution V6n(t) when t varies in the interval [0,20] ms. To do so, we will use the equivalent resistance calculated before.

We start with the equation of a linear capacitor:

\begin{gather}
	Q(t)=Cv(t) \iff Q(t)=-CR_{eq}i(t) \iff Q(t)=-CR_{eq}\frac{dQ(t)}{dt} \iff \frac{dQ(t)}{Q(t)}=-\frac{dt}{CR_{eq}} \iff \nonumber \\ \iff \int_{Q_0}^{Q(t)}\frac{1}{Q(t)}dQ(t)=\int_{0}^{t}-\frac{1}{CR_{eq}}dt \iff ln(\frac{Q(t)}{Q_0})=-\frac{t}{CR_{eq}} \iff e^{ln(\frac{Q(t)}{Q_0})}=e^{-\frac{t}{CR_{eq}}} \iff \nonumber \\ \iff \frac{Q(t)}{Q_0}=e^{-\frac{t}{CR_{eq}}} \iff Q(t)=Q_0e^{-\frac{t}{CR_{eq}}} \iff \frac{Q(t)}{C}=\frac{Q_0}{C}e^{-\frac{t}{CR_{eq}}} \iff \nonumber \\ \iff V_{6n}(t)=V_0e^{-\frac{t}{CR_{eq}}}
\end{gather}

Assuming the initial condition of $V_0=V_x$ the natural solution for the capacitor is:

\vspace{10pt}

$V_{6n}(t)=V_xe^{-\frac{t}{CR_{eq}}}$

\vspace{10pt}

We can now plot the graph of the solution for t $\in$[0,20] ms:

\begin{figure}[h]
	\centering
	\includegraphics[width=400pt, trim={0, 7cm, 0, 6cm}, clip]{natural.pdf}
	\caption{Natural solution of the capacitor}
	\label{fig:natural}
\end{figure}

\newpage

