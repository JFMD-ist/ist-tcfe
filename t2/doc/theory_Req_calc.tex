\hspace{12pt} In order to calculate the equivalent resistance seen from the terminals of the capacitor, we use the Thévenin's Theorem by replacing the capacitor with a voltage source with an equivalent value of Vx = V(6)-V(8) = \input{Vx} V, this is done in order to simulate the capacitor being fully charged. Now the equivalent resistance is the value of the hypotetical resistor in series with the voltage source Vx. We now assume Vs = 0V, and since the voltage source has an interior resistance of 0, it is equivalent to short-circuit that source. We can now perform a node analysis using KCL and other restrains to determine the voltages in each node, and then the currents in each mesh.

Equations (1), (2), (4), (5), (6) and (7) from the previous section are equivalent in this one. We can get the remaining 2 equations from:

\begin{gather}
	V_x=V_6-V_8
\end{gather}

And from node 4, which has been short-circuited with node 1:

\begin{gather}
	-I_1+I_4-I_6=0 \iff \nonumber \\
	\iff -G_1(V_1-V_2)+G_4(V_5-V_4)-G_6(V_4-V_7)=0 \iff \\
	\iff -G_1V_1+G_1V_2-(G_4+G_6)V_4+G_4V_5+G_6V_7=0 \nonumber
\end{gather}

With all the equations we can now form the following linear equation system:

\fontsize{8}{12}\selectfont
$\begin{bmatrix}
    G1 & -(G1+G2+G3) & G2 & 0 & G3 & 0 & 0 & 0 \\
    0 & Kb+G2 & -G2 & 0 & -Kb & 0 & 0 & 0 \\
    0 & 0 & 0 & 0 & 0 & 1 & 0 & -1 \\
    0 & 0 & 0 & G6 & 0 & 0 & -(G6+G7) & G7 \\
    0 & 0 & 0 & 1 & 0 & 0 & 0 & 0 \\
    1 & 0 & 0 & -1 & 0 & 0 & 0 & 0 \\
    0 & 0 & 0 & -KdG6 & -1 & 0 & KdG6 & 1 \\
    -G1 & G1 & 0 & -(G4+G6) & G4 & 0 & G6 & 0
\end{bmatrix}$ $\cdot$
$\begin{bmatrix}
     V_1 \\V_2 \\V_3 \\V_4 \\V_5 \\V_6 \\V_7 \\V_8
\end{bmatrix}$ =
$\begin{bmatrix}
    0 \\0 \\Vx \\0 \\0 \\0 \\0 \\0 
\end{bmatrix}$

\fontsize{11}{12}\selectfont
\vspace{20pt}

Using Octave to calculate the solution of the system we obtain:

\begin{figure}[h]
	\begin{center}
	    \begin{minipage}{.3\textwidth}
		\flushright
		\begin{tabular}{|c|}
		    \hline
		    $V_1$ \\
		    \hline
		    $V_2$ \\
		    \hline
		    $V_3$ \\
		    \hline
		    $V_4$ \\
		    \hline
		    $V_5$ \\
		    \hline
		    $V_6$ \\
		    \hline
		    $V_7$ \\
		    \hline
		    $V_8$ \\
		    \hline
		\end{tabular}
	    \end{minipage}
	    \hspace{-8pt}
	    \begin{minipage}{.3\textwidth}
		\flushleft
		\input{voltages2.tex}
	    \end{minipage}
	\end{center}
	\caption{Electric potential of the 8 nodes}
	\label{theory_voltages}
\end{figure}

Now that we have the values for the voltages in every node, we can use Ohm's Law to calculate the current flowing through $V_x$:
$V_6 - V_8 = R_5 \cdot I_x \iff I_x = \frac{V_6 - V_8}{R_5} = $ \input{Ix.tex}A,
and with the current $I_x$, the equivalent resistance becomes $R_{eq} = \frac{V_x}{I_x} =$ 3140.3333$\Omega$.
