\hspace{12pt} Now we will analyse the forced solution v6f(t) when t varies in [0,20] ms. To do so we will do once more nodal analysis. Using the phasor $\widetilde{V_S}=V_se^{-j\varphi_s}$, and using equations (1), (2), (4), (5), (6) and (7), we can now get the final 2 equations from:

Node 6:

\begin{gather}
	-\widetilde{I_b}+\widetilde{I_5}-\widetilde{I_c}=0 \iff \nonumber \\
    \iff -K_b(\widetilde{V_2}-\widetilde{V_5})+G_5(\widetilde{V_5}-\widetilde{V_6})-Y_c(\widetilde{V_6}-\widetilde{V_8})=0 \iff \\
    \iff  Kb\widetilde{V_2}-(Kb+G5)\widetilde{V_5}+(Yc+G5)\widetilde{V_6}-Yc\widetilde{V_8}=0 \nonumber
\end{gather}

Super Node:

\begin{gather}
	\widetilde{I_3}-\widetilde{I_4}-\widetilde{I_5}+\widetilde{I_7}+\widetilde{I_c}=0 \iff \nonumber \\
    \iff G_3(\widetilde{V_2}-\widetilde{V_5})-G_4(\widetilde{V_5}-\widetilde{V_4})-G_5(\widetilde{V_5}-\widetilde{V_6})+G_7(\widetilde{V_7}-\widetilde{V_8})+Y_c(\widetilde{V_6}-\widetilde{V_8})=0 \iff \\
    \iff G3\widetilde{V_2}+G4\widetilde{V_4} -(G3+G4+G5)\widetilde{V_5}+(G5+Yc)\widetilde{V_6}+G7\widetilde{V_7}-(G7+Yc)\widetilde{V_8}=0 \nonumber
\end{gather}

With the 8 equations we have the following linear equation system:

\vspace{10pt}

\fontsize{8}{12}\selectfont
$\begin{bmatrix}
    G1 & -(G1+G2+G3) & G2 & 0 & G3 & 0 & 0 & 0 \\
    0 & Kb+G2 & -G2 & 0 & -Kb & 0 & 0 & 0 \\
    0 & Kb & 0 & 0 & -Kb-G5 & Yc+G5 & 0 & -Yc \\
    0 & 0 & 0 & G6 & 0 & 0 & -(G6+G7) & G7 \\
    0 & 0 & 0 & 1 & 0 & 0 & 0 & 0 \\
    1 & 0 & 0 & -1 & 0 & 0 & 0 & 0 \\
    0 & 0 & 0 & -KdG6 & -1 & 0 & KdG6 & 1 \\
    0 & G3 & 0 & G4 & -G3-G4-G5 & G5+Yc & G7 & -G7-Yc
\end{bmatrix}$ $\cdot$
$\begin{bmatrix}
     V_1 \\V_2 \\V_3 \\V_4 \\V_5 \\V_6 \\V_7 \\V_8
\end{bmatrix}$ =
$\begin{bmatrix}
    0 \\0 \\0 \\0 \\0 \\1 \\0 \\0 
\end{bmatrix}$

\fontsize{11}{12}\selectfont
\vspace{20pt}
Using Octave to calculate the solution of the system we obtain:

\begin{figure}[h]
	\begin{center}
	    \begin{minipage}{.3\textwidth}
		\flushright
		\begin{tabular}{|c|}
		    \hline
		    $V_1$ \\
		    \hline
		    $V_2$ \\
		    \hline
		    $V_3$ \\
		    \hline
		    $V_4$ \\
		    \hline
		    $V_5$ \\
		    \hline
		    $V_6$ \\
		    \hline
		    $V_7$ \\
		    \hline
		    $V_8$ \\
		    \hline
		\end{tabular}
	    \end{minipage}
	    \hspace{-8pt}
	    \begin{minipage}{.3\textwidth}
		\flushleft
		\input{v3_mag.tex}
	    \end{minipage}
	\end{center}
	\caption{Electric potential of the 8 nodes}
	\label{theory_voltages}
\end{figure}

\newpage
