\label{simulation}
\subsection{Operating Point $t < 0$}
\hspace{12pt} Using Ngspice to perform an operating point analysis for $t < 0$, we obtain the following results for the voltages and the currents:
\begin{figure}[h]
	\begin{minipage}{.45\textwidth}
		\begin{center}
			\begin{tabular}{|c|c|}
				\hline
				\textbf{Name} & \textbf{Value (A or V)} \\
				\hline
				\input{op_1_tab.tex}
			\end{tabular}
		\end{center}
		\caption{Current and voltage values obtained from the ngspice simulation}
		\label{op1_results}
	\end{minipage}
	\begin{minipage}{.45\textwidth}
		\begin{center}
			\begin{tabular}{|c|c|}
				\hline
				\textbf{Name} & \textbf{Value (A or V)} \\
				\hline
				\input{op_2_tab.tex}
			\end{tabular}
		\end{center}
		\caption{Current and voltage values obtained from the ngspice simulation}
		\label{op2_results}
	\end{minipage}
\end{figure}
\fontsize{11}{12}\selectfont


\subsection{Operating Point $v_S(t) = 0$}
\hspace{12pt} After setting $v_S = 0$ and replacing the capacitor with a voltage source $V_x = V_6 - V_8$ (with $V_6$ and $V_8$ obtained from the nodal analysis), we obtain the following results for the voltages and the currents: 

