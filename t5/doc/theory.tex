\subsection{Mesh Analysis and Central Frequency Results}
\hspace{12pt} For the theoretical analysis, the OP-AMP model used in the lectures was used, replacing this component with an input and output impedance components and a voltage controlled voltage source. Using this model and applying KVL in the meshes of the circuit, the following equations were obtained:

\vspace{20pt}
\begin{equation}	
    \begin{cases}
        -Vin+I_AZ_{C1}+R_1(I_A+I_B)=0 \\
        R_1(I_A+I_B)+Z_{in}(I_B+I_C)+R_2I_B=0 \\
        Z_{in}(I_B+I_C)+R_3I_C+Z_{out}(I_C-I_D)+A_VZ_{in}(I_B+I_C)=0 \\
        -A_VZ_{in}(I_D+I_C)+Z_{out}(I_D-I_C)+R_4I_D+Z_{C2}I_D=0
    \end{cases}
    \iff \\
\end{equation}
\vspace{20pt}
\begin{equation}
    \iff
    \begin{cases}
        (Z_{C1}+R_1)I_A+R_1I_B=V_{in} \\
        R_1I_A+(R_1+R_2+Z_{in})I_B+Z_{in}I_C=0 \\
        (Z_{in}+A_VZ_{in})I_B+(Z_{in}+R_3+Z_{out}+A_VZ_{in})I_C-Z_{out}I_D=0 \\
        -A_VZ_{in}I_B+(-A_VZ_{in}-Z_{out})I_C+(Z_{out}+R_4+Z_{C2})I_D=0
    \end{cases}
    \nonumber
\end{equation}
\vspace{30pt}

Where $V_{in}$ represents the input voltage phasor ,$Z_{C1}$ represents the impendance of capacitor $C_1$, $Z_1$ the impedance of resistor $R_3$, $Z_2$ the feedback impedance, $Z_{C2}$ the impedance of capacitor $C_2$, $Z_{in}$ and $Z_{out}$ the input and output impedance of the OP-AMP, and $A_v$ the gain of the OP-AMP.
Using Octave to solve the system of equations, it is easy to obtain $V_{out}$, simply using $V_{out}=I_D \cdot Z_{C2}$. In order to obtain the input and output impedance of the filter we can simply take the correponding voltages and divide them by the current flowing across them, we thus have $Z_{input} = \frac{V_{in}}{I_A}$ and $Z_{output} = \frac{V_{out}}{I_D}$. The gain is calculated as the ratio of the output and input voltages $\frac{V_{out}}{V_{in}}$.
\vspace{30pt}

\begin{figure}[h]
	\centering
	\input{central_freq.tex}
	\caption{Octave mesh analysis results calculated at the intended central frequency 1kHZ}
	\label{fig:central_th}
\end{figure}

As we can see, the resulting gain value is very close to the intended 40dB, this value will be further compared to the simulation result in Section \ref{sec:conc}.

\pagebreak

\subsection{Frequency Response}
\hspace{12pt} Due to the complex nature of the impedances $Z_{C1}$ and $Z_{C2}$, the gain of the circuit is dependent of the frequency at which it is analyzed, by creating a frequency vector from 10Hz to 100MHz and evaluating the result obtained in the mesh analysis described above for each frequency point, we can obtain the frequency response of the filter. Given the complex nature of this results we have 2 plots, a magnitude (Figure \ref{fig:freq_th}a) and a phase plot (Figure \ref{fig:freq_th}b).

\vspace{10pt}
\begin{figure}[h]
	\centering
	\subfigure[]{\includegraphics[width=0.45\textwidth, trim={0, 6cm, 0, 6cm}, clip]{gain_mag.pdf}}
	\subfigure[]{\includegraphics[width=0.45\textwidth, trim={0, 6cm, 0, 6cm}, clip]{gain_ph.pdf}}
	\caption{(a) Magnitude plot (b) Phase plot of the frequency response output}
	\label{fig:freq_th}
\end{figure}

\vspace{10pt}
We can clearly see the nature of the bandpass in Figure \ref{fig:freq_th}a, the orange line present in the magnitude plot represents the cutoff value, it is obtained by subtracting 3dB to the maximum value given by the graph and its two intersections with $\frac{V_{out}}{V_{in}}(f)$ gives us the two cutoff frequencies of the filter circuit, the central frequency can then be obtained as a geometric mean of these two values: 
$f_{central} = \sqrt{f_{low} f_{high}}$. By computing these values we arrive to the following results:

\vspace{10pt}
\begin{figure}[h]
	\centering
	\input{freq_response.tex}
	\caption{Octave frequency response results}
	\label{fig:freq_resp}
\end{figure}

Analyzing these results we see the central frequency is very close to the intended 1kHz. We can now define the \textit{Merit figure} as $M = \frac{1}{cost (frequency deviation + gain deviation + 10^{-6})}$ and considering the cost of the components used ($316.66MU$ for the resistors and capacitors and $13323.2920387MU$ for the OP-AMP) we get a merit figure of roughly $1.02 \cdot 10^{-6}$.

\pagebreak
