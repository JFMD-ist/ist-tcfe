\hspace{12pt} In this \textbf{\textit{final}} laboratory session we were asked to analyze a circuit composed of an OP-AMP and other components, with the objective of creating a bandpass filter with a central frequency of 1kHz	 and a gain of 40dB. After many iterations, we reached the final architecture of the circuit which can be seen in its general form in Figure \ref{fig:bandpass_amp} and in its full detailed form in Figure \ref{fig:circuit}.

\vspace{-10pt}
\begin{figure}[h]
 	\centering
	\includegraphics[width=.5\textwidth, trim={0 2cm 0 5cm}, clip]{Bandpass_amplifier.pdf}
	\vspace{-20pt}
	\caption{Bandpass amplifier circuit general architecture}
	\label{fig:bandpass_amp}
\end{figure}

\vspace{-25pt}
\begin{figure}[h]
 	\centering
	\includegraphics[width=.5\textwidth, trim={0 2cm 0 3cm}, clip]{Circuit_diagram.pdf}
	\vspace{-30pt}
	\caption{Bandpass amplifier circuit architecture}
	\label{fig:circuit}
\end{figure}

In the theoretical analysis we used a model (Figure \ref{fig:OP-AMP}) to aproximate the OP-AMP component and used it, alongside the Octave software, to calculate the currents in the different meshes of the circuit, obtaining this way the input and output impedances of the circuit, and also the theoretical frequency response of the filter.
\vspace{10pt}

After that, in the simulation section of the report, we simulated the circuit using the Ngspice software and performed an extensive performance analysis in order to achieve several goals which were set, this lead to the creation of several Octave scripts which aided us in finding the best possible values for the resistors and capacitors whilst staying within the limits of the components set by the professor. The first of these ("random\_gen.m") creates a random resistor and capacitor using a maximum number of each component set by the user, by creating a number of branches and assigning to each of them a random number of resistors of each type, this results in an equivalent resistor which can range from very simple and obvious to contruct to very complex stuctures which would be hard to come up with. The other two scripts - "resistor\_calc.m" and "capacitor\_calc.m" - take in a specific resistance (or capacitance) value and a maximum number of each component allowed and by running the same logic as the previous script 20000 times, checks the best possible combination to aproximate the requested value, after finding this best value it then output its resistance (or capacitance) as well as the structure generated in the form of a matrix whose columns represent the number of components of which type present in the branch and whose lines represent the number of branches of the structure.
\vspace{10pt}

Finally, in the conclusion we briefly recall the objectives of the lab session and conclude about the results obtained, and how they reflect the purpose of the laboratory session.

\pagebreak
