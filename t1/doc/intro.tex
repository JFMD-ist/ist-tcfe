\hspace{12pt} In this laboratorial session we analyzed an electrical circuit composed of four meshes, eight nodes and eleven total branches: 7 resistors (R1-R7), 1 voltage source (Va), 1 current source (Id), 1 voltage controlled current source (Ib) and, finally, 1 current controlled voltage source (Vc).

\begin{figure}[h]
	\centering
	\includegraphics[width=.5\textwidth]{Circuit_diagram.pdf}
	\caption{Circuit diagram}
	\label{fig:circuit}
\end{figure}

In section 2 we applied both the mesh and nodal methods, together with mathematical support from an octave script to solve the resulting systems of linear equations, analized the circuit theoretically. In section \ref{simulation}, after recreating the circuit using an Ngspice script and using an operating point analysis we arrived at the values for the currents in each mesh and the voltages in each node. 
Finally, in section \ref{comparison}, we used both of the previous results (theoretical and simulated) and by comparing them we reached our conclusion.

Below is the representation of the directions and names of the currents and nodes used throughout the assignment, along with the location of the GND that has been used in the simulation and the theoretical analysis.
\begin{figure}[h]
	\centering
	\includegraphics[width=.4\textwidth]{Circuit_specs.pdf}
	\caption{Node names and current directions}
	\label{fig:circuit_spec}
\end{figure}

\pagebreak
