\label{simulation}
\hspace{12pt} Using the software Ngspice, we recreated the circuit (Figure ~\ref{fig:circuit}) with the specifications as seen in Figure ~\ref{fig:circuit_spec} and after performing an \textbf{operational point analysis} (.op in the ngspice script) obtained the folowing results:

\begin{figure}[h]
	\begin{center}
		\begin{tabular}{|c|c|}
			\hline
			\textbf{Name} & \textbf{Value (A or V)} \\
			\hline
			\input{op_tab.tex}
		\end{tabular}
	\end{center}
	\caption{Current and voltage values obtained from the ngspice simulation}
	\label{sim_results}
\end{figure}

Where the @ symbol indicates a current flowing through a component (ie @r1 means the current flowing through resistor 1 with sign according to the defined polarity of R1 and the assumed current direction) and a single lowercase letter indicates the voltage in that node. It's important to note that the voltage in node A is not present in the simulation as it was defined to be the GND node ($V_A = 0V$)and thus will not be printed by the script.


