\hspace{12pt} This circuit is composed of 8 nodes, which have been named in the image in the beginning of the document (Figure ~\ref{fig:circuit_spec}). The Kirchhoff's Current Law says that the sum of the current that enters a node, with the current that leaves a node, must be zero. However this law cannot be applied in nodes connected to voltage sources, so we can only use this law in 4 of the 8 nodes.

We will now analyze the 4 nodes using Kirchhoff's Current Law.


\vspace{10pt}
Node C:
\begin{gather}
    I_1+I_2-I_3=0 \iff \nonumber \\ 
    \iff G_1(V_B-V_C)+G_2(V_D-V_C)-G_3(V_C-V_H)=0 \iff \\
    \iff G_1V_B-(G_1+G_2+G_3)V_C+G_2V_D+G_3V_H \nonumber
\end{gather}

Node D:
\begin{gather}
    -I_2+I_b=0 \iff \nonumber \\
    \iff -G_2(V_D-V_C)+K_b(V_C-V_H) \iff \\
    \iff (K_b+G_2)V_C-G_2V_D-K_bV_H=0 \nonumber
\end{gather}

Node E:
\begin{gather}
    -I_b-I_5+I_d=0 \iff \nonumber \\
    \iff -K_b(V_C-V_H)-G_5(V_E-V_H)+I_d=0 \iff \\
    \iff K_bV_C+G_5V_E-(K_b+G_5)V_H=I_d \nonumber
\end{gather}

Node G:
\begin{gather}
    I_c-I_7=0 \iff \nonumber \\
    \iff G_6(0-V_G)-G_7(V_G-V_F)=0 \iff \\
    \iff -(G_6+G_7)V_G+G_7V_F=0 \nonumber
\end{gather}
\vspace{5pt}

Assuming node A has 0 potential, we can conclude by observation:
\begin{equation}
    V_B=V_a
\end{equation}
\vspace{5pt}

The voltage controlled voltage source also gives:
\begin{gather}
    V_c=K_cI_c \iff \nonumber \\
    \iff V_H-V_F=K_cG_6(0-V_G) \iff \\
    \iff -V_F+K_cG_6V_G+V_H=0 \nonumber
\end{gather}
\vspace{5pt}

We get the last equation using the Ohm's Law in resistor 4:
\begin{gather}
    V_H-0=R_4(I_A+I_C) \iff \nonumber \\
    \iff V_H=R_4(G1(V_B-V_C)+G_6(0-V_G)) \iff \\
    \iff R_4G_1V_B-R_4G_1V_C-R_4G_6V_G-V_H=0 \nonumber
\end{gather}

Using the 7 equations above, we have the following linear equations system:
\vspace{10pt}

\fontsize{8}{12}\selectfont
$\begin{bmatrix}
    G1 & -(G1+G2+G3) & G2 & 0 & 0 & 0 & G3 \\
    0 & Kb+G2 & -G2 & 0 & 0 & 0 & -Kb \\
    0 & Kb & 0 & G5 & 0 & 0 & -(Kb+G5) \\
    0 & 0 & 0 & 0 & G7 & -(G6+G7) & 0 \\
    1 & 0 & 0 & 0 & 0 & 0 & 0 \\
    0 & 0 & 0 & 0 & -1 & KcG6 & 1 \\
    R4G1 & -R4G1 & 0 & 0 & 0 & -R4G6 & -1
\end{bmatrix}$ $\cdot$
$\begin{bmatrix}
     V_B \\V_C \\V_D \\V_E \\V_F \\V_G \\V_H
\end{bmatrix}$ =
$\begin{bmatrix}
    0 \\0 \\I_d \\0 \\V_a \\0 \\0
\end{bmatrix}$

\fontsize{11}{12}\selectfont
\vspace{20pt}
Using Octave to calculate the solution of the system we obtain:

\begin{figure}[h]
	\begin{center}
	    \begin{minipage}{.3\textwidth}
		\flushright
		\begin{tabular}{|c|}
		    \hline
		    $V_A$ \\
		    \hline
		    $V_B$ \\
		    \hline
		    $V_C$ \\
		    \hline
		    $V_D$ \\
		    \hline
		    $V_E$ \\
		    \hline
		    $V_F$ \\
		    \hline
		    $V_G$ \\
		    \hline
		    $V_H$ \\
		    \hline
		\end{tabular}
	    \end{minipage}
	    \hspace{-8pt}
	    \begin{minipage}{.3\textwidth}
		\flushleft
		\input{voltages.tex}
	    \end{minipage}
	\end{center}
	\caption{Electric potential of the 8 nodes}
	\label{theory_voltages}
\end{figure}

We can now use Ohm's Law to calculate the current flowing through each resistor which, consequently, allows us to reach the values for the currents in the 4 meshes.
