\documentclass[11pt]{article}
\usepackage[utf8]{inputenc}
\usepackage[lmargin=2cm, tmargin=3cm, rmargin=2cm, bmargin=2cm, headheight=5cm]{geometry}
\usepackage{graphicx}
\usepackage{subfigure}
\usepackage{enumerate}
\usepackage{listings}
\usepackage{caption}
\usepackage{float}
\usepackage{hyperref}
\usepackage{fancyhdr}
\usepackage{amsmath}

\pagestyle{fancy}{
\rhead{\includegraphics[width=30mm]{../other/logo_ist.png}}
\lhead{TCFE lab report}
}


\begin{document}
\thispagestyle{empty}
\begin{figure}[h]
	\centering
	\includegraphics[width=.6\textwidth]{../../figlib/IST_A_CMYK_POS.pdf}
\end{figure}

\begin{center}
	\huge{Circuit Theory and Electronic Fundamentals}
	
	\huge{Lab Report - T1}
	
	\vspace{30pt}
	
	\large{Professor: José Sousa}
	
	\vspace{20pt}
	
	\large{95801 - João Domingos}
	
	\large{96382 - Francisco Cadavez}
	
	\large{97087 - Miguel Fernandes}
\end{center}

\pagebreak

\tableofcontents

\pagebreak

\section{Introduction}
\hspace{12pt} In this laboratorial session we were tasked to create an AC/DC converter circuit (whose basic layout is described in Figure \ref{fig:ac_dc}), this includes a transformer, a full-wave rectifier an envelope detector and a voltage regulator circuits. The specific arquitecture for these sub-circuits is described in Figure \ref{fig:circuit}.

\begin{figure}[h]
	\centering
	\includegraphics[width=.5\textwidth, trim={0 4cm 0 4cm}, clip]{AC_DC.pdf}
	\caption{AC/DC converter circuit layout}
	\label{fig:ac_dc}
\end{figure}
 
 \begin{figure}[h]
 	\centering
	\includegraphics[width=.5\textwidth, trim={0 4cm 0 4cm}, clip]{Circuit_diagram.pdf}
	\caption{AC/DC converter circuit arquitecture specifications}
	\label{fig:circuit}
 \end{figure}
 
For the theoretical analysis (section \ref{sec:theory}), we began with the envelope detector where we used Octave to calculate the instant at which the diode switched off and used it as a reference to plot the output together with the equations given in the lesson 14. The voltage regulator output required the use of KVL along with the Diode equation (given in the slides of lecture 11) to get a non-linear equation which we then solved using the Newton-Raphson method for each instant of the time interval considered.

In the simulation section, we recreated the circuit layout in Ngspice, replacing the transformer with a model using 2 entangled dependent sources, and performed a transient analysis over 10 periods of the input signal, plotting the voltage output at the envelope detector and voltage regulator portions of the circuit.

It is important to note that since an AC/DC converter cannot ever be perfect, there exist some slight imperfections in the output signal: oscillations persist though with a much smaller amplitude, and the signals do not oscillate perfectly around V = 12V. These imperfections (named ripple and deviation respectively) are used along with the cost of the components (1MU (Monetary Unit) per $k \Omega$ in the resistors, 1MU per $\mu F$ in the capacitor, and 0.1MU per diode) to obtain the merit figure (defined ahead in the theoretical analysis section). The Octave and Ngspice scripts are coded to calculate these values automatically.

Finally, in section \ref{sec:conc}, we compared the plots of the results of the different sections and concluded that the models of the diodes used in the different scripts are most likely the cause of the disagreement between plots.

\pagebreak


\pagebreak
\section{Theoretical Analysis}

\subsection{Mesh Analysis Method}
\hspace{12pt} The circuit is composed of 4 meshes. The mesh analysis method tells us that the current inside a mesh is constant, and therefore, we can use Kirchhoff's Voltage Law in each mesh to determine the current.

The four elemental meshes in the circuits are: mesh A, composed by the nodes 0, B, C and H; mesh B, composed by the nodes C, D, E and H; mesh C, composed by the nodes 0, H, G and F; and mesh D, composed by the nodes H, E and F. We define $I_A$, $I_B$, $I_C$ and $I_D$ as the currents inside meshes A, B, C and D, respectively. 

We will now analyze each individual mesh, using Kirchhoff's Voltage Law.
\vspace{10pt}

Mesh A:
\begin{equation}
    \begin{split}
        R_1I_A+R_3(I_A+I_B)+R_4(I_A+I_C)=V_a \iff \\
        \iff (R_1+R_3+R_4)I_A+R_3I_B+R_4I_C=V_a
    \end{split}
\end{equation}

Mesh B:
\begin{equation}
    \begin{split}
        I_B=K_bV_B \iff I_B=K_bR_3(I_A+I_B) \iff \\
        \iff K_bR_3I_A+(K_BR_3-1)I_B=0
    \end{split}
\end{equation}

Mesh C:
\begin{equation}
    \begin{split}
        R_4(I_A+I_C)+R_6I_C+R_7I_C+V_C=0 \iff \\
        \iff R_4I_A+(R_4+R_6+R_7+K_c)I_C=0
    \end{split}
\end{equation}

Mesh D:
\begin{equation}
    I_D=I_d  \hspace{30pt} \textit{(by observation)}
\end{equation}

\vspace{20pt}
With the 4 equations above, we can form the following linear equation system:
\vspace{20pt}

$\begin{bmatrix}
    R_1 + R_3 + R_4 & R_3        & R_4             & 0 \\
    K_bR_3          & K_bR_3 - 1 & 0               & 0 \\
    R_4             & 0          & R_4+R_6+R_7-K_c & 0 \\
    0               & 0          & 0               & 1 
\end{bmatrix}$ $\cdot$
$\begin{bmatrix}
     I_A \\ I_B \\I_C \\I_D
\end{bmatrix}$ =
$\begin{bmatrix}
    V_a \\ 0 \\ 0 \\ I_d
\end{bmatrix}$

\vspace{20pt}
Using Octave to calculate the solution of the system we obtain:
\vspace{20pt}

\begin{figure}[h]
	\begin{center}
	    \begin{minipage}{.3\textwidth}
		\flushright
		\begin{tabular}{|c|}
		    \hline
		    $I_A$ \\
		    \hline
		    $I_B$ \\
		    \hline
		    $I_C$ \\
		    \hline
		    $I_D$ \\
		    \hline
		\end{tabular}
	    \end{minipage}
	    \hspace{-8pt}
	    \begin{minipage}{.3\textwidth}
		\flushleft
		\input{currents.tex}
	    \end{minipage}
	\end{center}
	\caption{Current values for the different meshes}
	\label{theory_currents}
\end{figure}

 Knowing the currents in all the circuit meshes, we can use Ohm's Law in each resistor to calculate the voltage drop and thus extrapolate the values for all of the node voltages.
 \newpage
 


\subsection{Node Analysis Method}
\hspace{12pt} This circuit is composed of 8 nodes, which have been named in the image in the beginning of the document (Figure ~\ref{fig:circuit_spec}). The Kirchhoff's Current Law says that the sum of the current that enters a node, with the current that leaves a node, must be zero. However this law cannot be applied in nodes connected to voltage sources, so we can only use this law in 4 of the 8 nodes.

We will now analyze the 4 nodes using Kirchhoff's Current Law.


\vspace{10pt}
Node C:
\begin{gather}
    I_1+I_2-I_3=0 \iff \nonumber \\ 
    \iff G_1(V_B-V_C)+G_2(V_D-V_C)-G_3(V_C-V_H)=0 \iff \\
    \iff G_1V_B-(G_1+G_2+G_3)V_C+G_2V_D+G_3V_H \nonumber
\end{gather}

Node D:
\begin{gather}
    -I_2+I_b=0 \iff \nonumber \\
    \iff -G_2(V_D-V_C)+K_b(V_C-V_H) \iff \\
    \iff (K_b+G_2)V_C-G_2V_D-K_bV_H=0 \nonumber
\end{gather}

Node E:
\begin{gather}
    -I_b-I_5+I_d=0 \iff \nonumber \\
    \iff -K_b(V_C-V_H)-G_5(V_E-V_H)+I_d=0 \iff \\
    \iff K_bV_C+G_5V_E-(K_b+G_5)V_H=I_d \nonumber
\end{gather}

Node G:
\begin{gather}
    I_c-I_7=0 \iff \nonumber \\
    \iff G_6(0-V_G)-G_7(V_G-V_F)=0 \iff \\
    \iff -(G_6+G_7)V_G+G_7V_F=0 \nonumber
\end{gather}
\vspace{5pt}

Assuming node A has 0 potential, we can conclude by observation:
\begin{equation}
    V_B=V_a
\end{equation}
\vspace{5pt}

The voltage controlled voltage source also gives:
\begin{gather}
    V_c=K_cI_c \iff \nonumber \\
    \iff V_H-V_F=K_cG_6(0-V_G) \iff \\
    \iff -V_F+K_cG_6V_G+V_H=0 \nonumber
\end{gather}
\vspace{5pt}

We get the last equation using the Ohm's Law in resistor 4:
\begin{gather}
    V_H-0=R_4(I_A+I_C) \iff \nonumber \\
    \iff V_H=R_4(G1(V_B-V_C)+G_6(0-V_G)) \iff \\
    \iff R_4G_1V_B-R_4G_1V_C-R_4G_6V_G-V_H=0 \nonumber
\end{gather}

Using the 7 equations above, we have the following linear equations system:
\vspace{10pt}

\fontsize{8}{12}\selectfont
$\begin{bmatrix}
    G1 & -(G1+G2+G3) & G2 & 0 & 0 & 0 & G3 \\
    0 & Kb+G2 & -G2 & 0 & 0 & 0 & -Kb \\
    0 & Kb & 0 & G5 & 0 & 0 & -(Kb+G5) \\
    0 & 0 & 0 & 0 & G7 & -(G6+G7) & 0 \\
    1 & 0 & 0 & 0 & 0 & 0 & 0 \\
    0 & 0 & 0 & 0 & -1 & KcG6 & 1 \\
    R4G1 & -R4G1 & 0 & 0 & 0 & -R4G6 & -1
\end{bmatrix}$ $\cdot$
$\begin{bmatrix}
     V_B \\V_C \\V_D \\V_E \\V_F \\V_G \\V_H
\end{bmatrix}$ =
$\begin{bmatrix}
    0 \\0 \\I_d \\0 \\V_a \\0 \\0
\end{bmatrix}$

\fontsize{11}{12}\selectfont
\vspace{20pt}
Using Octave to calculate the solution of the system we obtain:

\begin{figure}[h]
	\begin{center}
	    \begin{minipage}{.3\textwidth}
		\flushright
		\begin{tabular}{|c|}
		    \hline
		    $V_A$ \\
		    \hline
		    $V_B$ \\
		    \hline
		    $V_C$ \\
		    \hline
		    $V_D$ \\
		    \hline
		    $V_E$ \\
		    \hline
		    $V_F$ \\
		    \hline
		    $V_G$ \\
		    \hline
		    $V_H$ \\
		    \hline
		\end{tabular}
	    \end{minipage}
	    \hspace{-8pt}
	    \begin{minipage}{.3\textwidth}
		\flushleft
		\input{voltages.tex}
	    \end{minipage}
	\end{center}
	\caption{Electric potential of the 8 nodes}
	\label{theory_voltages}
\end{figure}

We can now use Ohm's Law to calculate the current flowing through each resistor which, consequently, allows us to reach the values for the currents in the 4 meshes.


\subsection{Method Comparison}
\hspace{12pt} We have now calculated both the currents and the voltages of the entire circuit however, having used 2 different methods does not guarantee us they agree with each other. In order to test said agreement we will analyze 2 resistors: $R_1$ and $R_5$. \vspace{12pt}

Using Ohm's Law on $R_1$ (and following the current convention we set) we hope to see that $V_B - V_C = R_1I_A$ and by carrying out said calculation we get that $V_B - V_C = 1.93585V$ and $R_1I_A = 1.93585V$. \vspace{12pt}

Let us now use the same method to analyze R5, since we are assuming the current in $R_5$ flows torward node $V_H$, we should see that $V_E - V_H = R_5(I_D - I_B)$. $V_E - V_H = 9.00903V$, $R_5(I_D - I_B) = 9.00907V$ this difference most likely comes from rounding errors within the Octave program but it correspondes to an immesurable error so we can say with certainty that both the nodal and mesh analysis methods agree with each other's results.

\pagebreak
\section{Simulation}
\label{simulation}
Using the software Ngspice, we recreated the circuit (Figure ~\ref{fig:circuit}) with the specifications as seen in Figure ~\ref{fig:circuit_spec} and after performing an \textbf{operational point analysis} (.op in the ngspice script) obtained the folowing results:

\begin{figure}[h]
	\begin{center}
		\begin{tabular}{|c|c|}
			\hline
			\textbf{Name} & \textbf{Value (A or V)} \\
			\hline
			\input{op_tab.tex}
		\end{tabular}
	\end{center}
	\caption{Current and voltage values obtained from the ngspice simulation}
	\label{sim_results}
\end{figure}




\pagebreak
\section{Conclusion}
\label{comparison}
In order to compare the different results, we calculated the relative error between the theoretical and the simulated values $\varepsilon = \frac{Value_{simulation} - Value_{theory}}{Value_{theory}}$.

By computing the $\varepsilon$ of the various values, we obtain the following:

\begin{figure}[h]
	\centering
	\begin{tabular}{|c|c|}
		\hline
		Relative Error    &  Value(percentage) \\
		\hline
		$\varepsilon_A$   &  -5.336e-5 \\
		\hline
		$\varepsilon_B$   &      0     \\
		\hline
		$\varepsilon_C$   &      0     \\
		\hline
		$\varepsilon_D$   &      0     \\
		\hline		
	\end{tabular}
	\caption{Relative error in the current results}
	\label{current_error}
\end{figure}

\vspace{30pt}

\begin{figure}[h]
	\centering
	\begin{tabular}{|c|c|}
		\hline
		Relative Error & Value(percentage)  \\
		\hline
		$\varepsilon_A$   &     0      \\
		\hline
		$\varepsilon_B$   &  3.9947e-5 \\
		\hline
		$\varepsilon_C$   &  9.7695e-5 \\
		\hline
		$\varepsilon_D$   &     0      \\
		\hline
		$\varepsilon_E$   &  3.2395e-4 \\
		\hline
		$\varepsilon_F$   &     0      \\
		\hline
		$\varepsilon_G$   &  1.3434e-4 \\
		\hline
		$\varepsilon_H$   &  2.9954e-5 \\
		\hline
	\end{tabular}
	\caption{Relative error in the voltage results}
	\label{voltage_error}
\end{figure}


\end{document}
